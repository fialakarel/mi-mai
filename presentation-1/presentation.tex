\documentclass{beamer}
%\usetheme{PaloAlto}
\usetheme{Boadilla}
\usecolortheme{crane}
%\mode<presentation>

\usepackage[utf8]{inputenc}
\usepackage[czech]{babel}

\usepackage{graphicx}
\usepackage{caption}
%\usepackage{subcaption}

\usepackage{pifont} % symboly \ding{}


\makeatletter
\def\Ldescription{%
  \@ifnextchar[{\beamer@testforospec}{\beamer@descdefault\beamer@descriptionwidth\@@Ldescription}%
}

\def\beamer@testforospec[{\@ifnextchar<{\beamer@scandefaultospec[}{\@Ldescription[}}%

\def\beamer@scandefaultospec[#1]{\def\beamer@defaultospec{#1}\Ldescription}

\def\@Ldescription[#1]{%
\setbox\beamer@tempbox=\hbox{\def\insertdescriptionitem{#1}
  \usebeamertemplate**{description item}}%
\beamer@descdefault\wd\beamer@tempbox\@@description%
}%

\def\@@Ldescription{%
  \beamer@descdefault35pt%
  \list
  {}
  {\labelwidth\beamer@descdefault\leftmargin2.8em\let\makelabel\beamer@Ldescriptionitem}%
  \beamer@cramped%
  \raggedright
  \beamer@firstlineitemizeunskip%
}

\def\endLdescription{\ifhmode\unskip\fi\endlist}
\long\def\beamer@Ldescriptionitem#1{%
  \def\insertdescriptionitem{#1}%
  \hspace\labelsep{\parbox[b]{\dimexpr\textwidth-\labelsep\relax}{%
        \usebeamertemplate**{description item}%
    }}}
\makeatother

%==============================================================================



\title[Dokumentace protokolu Speedtest]{Dokumentace protokolu Speedtest}
\subtitle[Prezentace přístupu k řešení]{Prezentace přístupu k řešení}
\author[Bc. Karel Fiala]{Bc. Karel Fiala}
\institute[ČVUT FIT]{
   Fakulta informačních technologií \\
   České vysoké učení technické v~Praze
 }
\date{\today}


\begin{document}

\begin{frame}
   \titlepage
\end{frame}


%\begin{frame}{Obsah prezentace}
%   \tableofcontents
%\end{frame}


%\section{Úvodem}\label{sec:uvod}
\section{Zadání}
\begin{frame}{Zadání}
Na základě zachyceného provozu pro výkonnostní test sítě programu Speedtest pro různě rychlá připojení od mobilního po 10 Gbps analyzujte a podrobně zdokumentujte použitý protokol, časový profil a objem testovacích dat v závislosti na naměřené propustnosti a vysvětlete případné anomálie.

\bigskip

\hfill \textbf{$\rightarrow$ analyzujte a zdokumentujte} \hfill \hfill

\end{frame}


\begin{frame}{Přístup k řešení}
\begin{enumerate}
\item Wireshark
\bigskip
\item speedtest-cli
\bigskip
\item Jak to funguje\\ $\rightarrow$ fork, detailní výpisy + Wireshark
\bigskip
\item Provést mnoho měření (až do 10 Gbps?)\\ $\rightarrow$ skript, logovat, uměle omezovat rychlost
\bigskip
\item Zjistit závislosti, vynést grafy, napsat dokumentaci
\end{enumerate}

\end{frame}


%\section{Závěr}\label{sec:zaver}
%\subsection{Shrnutí}\label{sec:zaver1}
%\subsection{Literatura}\label{sec:zaver2}
%\subsection{Poděkování}\label{sec:zaver3}
%\subsection{Otázky}\label{sec:zaver4}




%\begin{frame}{obr}
 %   \begin{figure}[h!]
%	    \centering
%	    \includegraphics[width=0.9\textwidth]{nic.jpg}
%	    \caption{obr}
%	    \label{obr:obr}
 %   \end{figure}
%\end{frame}


\end{document}
