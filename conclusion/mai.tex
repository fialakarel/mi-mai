\documentclass[12pt,a4paper]{article}
%\usepackage{epsf,epic,eepic,eepicemu}
%\documentstyle[epsf,epic,eepic,eepicemu]{article}

\usepackage[pdftex]{graphicx}
\usepackage[utf8]{inputenc} %kodovani znaku v textovem souboru
%\usepackage[T1]{fontenc} %kodovani znaku na vystupu
\usepackage[czech]{babel} %prizpusobeni jazyku, napr. deleni slov
%\usepackage{a4wide}

\usepackage[margin=2.5cm]{geometry}

\usepackage{url}
\usepackage{hyperref}
\usepackage{setspace}
\usepackage{wrapfig}

\onehalfspacing


\begin{document}
\title{Semestrální projekt MI-MAI\\ 2015/2016\\
Dokumentace protokolu Speedtest \\
\vspace{10px}}
\author{Karel Fiala \\
\vspace{10px} \\
\small České vysoké učení technické v~Praze\\
\small Fakulta informačních technologií\\
\small Thákurova 9, 160 00 Praha 6\\
\small Česká republika \\
\vspace{10px} \\
}
\date{\today}
\maketitle
\thispagestyle{empty}

% takhle se pise komentar

\clearpage
\thispagestyle{empty}
\tableofcontents
\clearpage

%\begin{abstract}
%
%Your abstract goes here...
%...
%\end{abstract}
%\clearpage


%\part*{část jedna}						% * zaridi ze to nebude cislovane
%\addcontentsline{toc}{part}{Úvod}  % toto přidá popisek do obsahu
%doplnit text
%\pagebreak


\section{Popis protokolu}

\subsection{Obecné}

Protokol speedtest operuje nad TCP (socket či HTTP), měří ping a rychlost oběma směry. Existuje několik implementací, které se od sebe i významně liší. Oficiální implementace využívá technologie Adobe Flash a pro měření využívá zejména TCP streamů. Nejznámější neoficiální implementace je napsaná v programovacím jazyku python a využívá HTTP fallback speedtest serverů. 


\subsection{Neoficiální verze -- program speedtest-cli}

Program speedtest-cli je jednoduchý python skript, který využívá speedtest servery a měří vůči nim rychlost klienta pomocí HTTP. Výhodou je jednoduchost řešení, které lze využít třeba i na serverech, kde není možné spustit prohlížeč a Adobe Flash. Kód skriptu je přehledný a dobře dokumentovaný. Nevýhodou může být lehce odlišná implementace měření od \textit{HTTP Legacy Fallback}, které v případě potřeby může využít aplikace v Adobe Flash.


\subsection{Oficiální verze -- Adobe Flash}

Verzi v Adobe Flash je uživatelsky přívětivá a snaží se využít TCP socketů. V případě selhání, je stále k dispozici HTTP Legacy Fallback přístup. Naměřené výsledky více reflektují skutečnost i přes vysokou zátěž procesoru způsobenou použitou technologií.

Verze pro HTML5, která již nepotřebuje Adobe Flash je stále ve fázi testování.


\subsection{Analýza protokolu}

\subsubsection{Průběh komunikace}

Popis průběhu komunikace pro program speedtest-cli. Oficiální verze přes TCP posílá binární data a také navazuje kontrolní spojení, které má zajistit efektivní využití pásma.

\begin{figure}[h]
%\begin{wrapfigure}{r}{0.4\textwidth}
\centering
\includegraphics[width=0.3\textwidth]{random350x350.jpg}
\caption{Ukázka stahovaného obrázku random350x350.jpg}
\label{random350}
\end{figure}


\begin{itemize}
\item Stáhne se seznam serverů z www.speedtest.net/speedtest-servers-static.php.
\item Servery se seřadí, vybere se pět nejbližších dle lokace.
\item Otestuje se latence přes <server>/speedtest/latency.txt.
\item Začne se testovat rychlost stahování pomocí stahování náhodných jpg obrázků s pevnou velikostí. Každý obrázek se stahuje 4x a stahuje se přes 6 streamů najednou.

Obrázky jsou o velikostech:
%350x350px , 500x500px, 750x750px, 1000x1000px, 1500x1500px, 2000x2000px, 2500x2500px, 3000x3000px, 3500x3500px a 4000x4000px.


\begin{itemize}
\item 240K random350x350.jpg
\item 494K random500x500.jpg
\item 1.1M random750x750.jpg
\item 1.9M random1000x1000.jpg
\item 4.3M random1500x1500.jpg
\item 7.6M random2000x2000.jpg
\item  12M random2500x2500.jpg
\item  17M random3000x3000.jpg
\item  24M random3500x3500.jpg
\item  31M random4000x4000.jpg
\end{itemize}

\item Po 10 sec se stahování ukončí a ze stažených dat se spočítá rychlost stahování.
\item Začne se testovat rychlost nahrávání dat (velká pole čísel) na adresu: <server>/speedtest/upload.php.
\item Po 10 sec se nahrávání ukončí a z odeslaných dat se spočítá rychlost nahrávání.
\end{itemize}

Maximálně se tedy stáhne 400MB dat. Stahování trvá maximálně 10sec, po které je linka plně saturována. Bod zlomu je tak 320mbit/s. Pod touto rychlostí bude stahování ukončeno časovačem dříve (přenese se i méně dat) a nad tuto rychlost bude stahování dokončeno dříve než za 10sec (více dat se nepřenese)


\section{Naměřené výsledky}


\begin{center}
\begin{tabular}{ | c || c || c | c | c | c |  }
\hline
umístění & klient   &  data TX & data RX & rychlost TX & rychlost RX	\\
\hline
\hline
FIT LAN (1gbit) & speedtest-cli & 26MB   & 430MB  & 250mbps   & 850mbps 	\\ \hline
FIT LAN (1gbit) & web           & 1060MB & 1600MB & 550mbps   & 780mbps 	\\ \hline
O2 ADSL (5mbit) & speedtest-cli & 2.2MB  & 22MB   & 0.15mbps  & 4.1mbps 	\\ \hline
O2 ADSL (5mbit) & web           & 1MB    & 9.5MB  & 0.3mbps   & 4.3mbps 	\\ \hline

\end{tabular}
\end{center}


\section{Závěr}

Každá implementace (Adobe Flash, Beta verze v HTML5, speedtest-cli, Android aplikace, iOS aplikace, \ldots) protokolu Speedtest se lehce liší a v každé verzi se mohou objevit různé anomálie dané použitou technologií. Princip a rozhraní poskytované servery však zůstává stejné.


\section{Zdroje a odkazy}

\subsection{Vlastní}

\begin{itemize}
\item \href{https://github.com/fialakarel/speedtest-cli/tree/mi-mai}
	    {https://github.com/fialakarel/speedtest-cli/tree/mi-mai}
\item \href{https://github.com/fialakarel/speedtest-cli/blob/mi-mai/speedtest\_cli.py}
	    {https://github.com/fialakarel/speedtest-cli/blob/mi-mai/speedtest\_cli.py}
\item \href{https://github.com/fialakarel/speedtest-cli/blob/mi-mai/speedtest-orig.log}
	    {https://github.com/fialakarel/speedtest-cli/blob/mi-mai/speedtest-orig.log}
\item \href{https://github.com/fialakarel/speedtest-cli/blob/mi-mai/speedtest-debug.log}
	    {https://github.com/fialakarel/speedtest-cli/blob/mi-mai/speedtest-debug.log}
\item \href{https://github.com/fialakarel/speedtest-cli/blob/mi-mai/speedtest-debug-all.log}
	    {https://github.com/fialakarel/speedtest-cli/blob/mi-mai/speedtest-debug-all.log}
\item \href{https://github.com/fialakarel/mi-mai/blob/master/presentation-2/random350x350.jpg}
	    {https://github.com/fialakarel/mi-mai/blob/master/presentation-2/random350x350.jpg}
\end{itemize}



\subsection{Externí}

\begin{itemize}
\item \href{https://support.speedtest.net/hc/en-us/articles/203845400-How-does-the-test-itself-work-How-is-the-result-calculated-}
	   {https://support.speedtest.net/}
\item \href{https://en.wikipedia.org/wiki/Speedtest.net}
	   {https://en.wikipedia.org/}
\item \href{https://www.ncta.com/platform/broadband-internet/how-do-internet-speed-tests-work-2/}
	   {https://www.ncta.com/}
\item \href{http://www.ookla.com/docs/UnderstandingBroadbandMeasurement.pdf}
	   {http://www.ookla.com/}
\end{itemize}



%\begin{itemize}
%\item seznam
%\end{itemize}


%% tabulka
%\begin{center}
%\begin{tabular}{ | c || c | c | c | c | }
%\hline
%N    &   čas	\\
%\hline
%\hline
%1000    &   0.20681 	\\ \hline
%2000    &   0.39094 	\\ \hline
%3000    &   0.82996 	\\ \hline
%4000    &   1.27902 	\\ \hline
%5000    &   2.92957 	\\ \hline
%6000    &   4.28045 	\\ \hline
%7000    &   7.42563 	\\ \hline
%8000    &   8.21492 	\\ \hline
%\end{tabular}
%\end{center}
%
%
%% obrázek
%\pagebreak
%\begin{figure}[h]
%\includegraphics[width=\textwidth]{graph/cuda/cuda1.png}
%\caption{Klasický multiplikativní algoritmus -- CUDA implementace}
%\label{cuda1}
%\end{figure}

% generujeme literaturu
% PRŮCHA, Jan a Soňa KOŤÁTKOVÁ. Předškolní pedagogika: učebnice pro střední a vyšší odborné školy. Vyd. 1. Praha: Portál, 2013, 181 s. ISBN 978-80-262-0495-4.
%\bibliographystyle{iso690}
%\bibliography{literatura}

%\appendix


\end{document}
