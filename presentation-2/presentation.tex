\documentclass{beamer}
%\usetheme{PaloAlto}
\usetheme{Boadilla}
\usecolortheme{crane}
%\mode<presentation>

\usepackage[utf8]{inputenc}
\usepackage[czech]{babel}

\usepackage{graphicx}
\usepackage{caption}
%\usepackage{subcaption}
\usepackage{verbatim}

\usepackage{pifont} % symboly \ding{}


\makeatletter
\def\Ldescription{%
  \@ifnextchar[{\beamer@testforospec}{\beamer@descdefault\beamer@descriptionwidth\@@Ldescription}%
}

\def\beamer@testforospec[{\@ifnextchar<{\beamer@scandefaultospec[}{\@Ldescription[}}%

\def\beamer@scandefaultospec[#1]{\def\beamer@defaultospec{#1}\Ldescription}

\def\@Ldescription[#1]{%
\setbox\beamer@tempbox=\hbox{\def\insertdescriptionitem{#1}
  \usebeamertemplate**{description item}}%
\beamer@descdefault\wd\beamer@tempbox\@@description%
}%

\def\@@Ldescription{%
  \beamer@descdefault35pt%
  \list
  {}
  {\labelwidth\beamer@descdefault\leftmargin2.8em\let\makelabel\beamer@Ldescriptionitem}%
  \beamer@cramped%
  \raggedright
  \beamer@firstlineitemizeunskip%
}

\def\endLdescription{\ifhmode\unskip\fi\endlist}
\long\def\beamer@Ldescriptionitem#1{%
  \def\insertdescriptionitem{#1}%
  \hspace\labelsep{\parbox[b]{\dimexpr\textwidth-\labelsep\relax}{%
        \usebeamertemplate**{description item}%
    }}}
\makeatother

%==============================================================================



\title[Dokumentace protokolu Speedtest]{Dokumentace protokolu Speedtest}
\subtitle[Prezentace stavu řešení]{Prezentace stavu řešení}
\author[Bc. Karel Fiala]{Bc. Karel Fiala}
\institute[ČVUT FIT]{
   Fakulta informačních technologií \\
   České vysoké učení technické v~Praze
 }
\date{\today}


\begin{document}

\begin{frame}
   \titlepage
\end{frame}


%\begin{frame}{Obsah prezentace}
%   \tableofcontents
%\end{frame}


%\section{Úvodem}\label{sec:uvod}
\section{Zadání}
\begin{frame}{Zadání}
Na základě zachyceného provozu pro výkonnostní test sítě programu Speedtest pro různě rychlá připojení od mobilního po 10 Gbps analyzujte a podrobně zdokumentujte použitý protokol, časový profil a objem testovacích dat v závislosti na naměřené propustnosti a vysvětlete případné anomálie.

\bigskip

\hfill \textbf{$\rightarrow$ analyzujte a zdokumentujte} \hfill \hfill 

\end{frame}


\begin{frame}{Stav řešení 1/2}
\begin{itemize}
\item \url{https://github.com/fialakarel/speedtest-cli/tree/mi-mai}
\medskip
\item Ukázka výstupu
\end{itemize}

\begin{quote}
Retrieving speedtest.net configuration...\\
Retrieving speedtest.net server list...\\
Testing from O2 Czech Republic (85.71.xxx.xxx)...\\
Selecting best server based on latency...\\
Hosted by Nej TV a.s. (Prague) [3.57 km]: 21.497 ms\\
Testing download speed\\
Download: 4.01 Mbit/s\\
Testing upload speed\\
Upload: 0.23 Mbit/s\\
\end{quote}

\end{frame}


\begin{frame}{Stav řešení 2/2}
\begin{itemize}
\item To samé se všemi debug výpisy má 9233 řádků \\
	\textit{\url{https://github.com/fialakarel/speedtest-cli/blob/mi-mai/speedtest-debug-all.log}}
\bigskip
\item Po odstranění přebytečných informací dostáváme 267 užitečných řádků,
	které prozradí, jak to celé funguje\\
	\textit{\url{https://github.com/fialakarel/speedtest-cli/blob/mi-mai/speedtest-debug.log}}
\end{itemize}

\end{frame}



\begin{frame}[fragile]{Ukázka zkráceného výstupu}
\begin{verbatim}
build_user_agent: (created) user_agent -> Mozilla/5.0 (Linux; U; 64bit; en-us) Python/2.7.11+ (KHTML, like Gecko) speedtest-cli/0.3.4
getConfig: downloaded ... parsing
closestServers: processing url -> ://www.speedtest.net/speedtest-servers-static.php
closesServers: closest server -> speedtest.vodafone.cz:8080
getBestServer: test latency for server -> http://speedtest.vodafone.cz/speedtest/latency.txt
getBestServer: latency -> 27.225
downloadSpeed: starting producer and consumer for 'q = Queue(6)'
downloadSpeed:producer: file -> http://rychlost.nejtv.cz/speedtest/random350x350.jpg
downloadSpeed: threads producer and consumer finished
uploadSpeed: starting producer and consumer for 'q = Queue(6)'
uploadSpeed:producer: size -> 250000
FilePutter.__init__: url -> http://rychlost.nejtv.cz/speedtest/upload.php start -> 1461432707.43 size -> 250000
uploadSpeed:consumer: len(finished) -> 0
\end{verbatim}
\end{frame}

\begin{frame}{Co víme}
\begin{itemize}
\item TCP
\item HTTP
\item random JPG -- nejdou komprimovat
\item pevně dané velikosti
\item 240kB - 30MB na obrázek
\item multithread
\item různé implementace
\item časovač
\end{itemize}

\begin{figure}[h!]
	    \centering
	    \includegraphics[width=1\textwidth]{random350x350.jpg}
	    \caption{obr}
	    \label{obr:obr}
\end{figure}

\end{frame}


%\section{Závěr}\label{sec:zaver}
%\subsection{Shrnutí}\label{sec:zaver1}
%\subsection{Literatura}\label{sec:zaver2}
%\subsection{Poděkování}\label{sec:zaver3}
%\subsection{Otázky}\label{sec:zaver4}




%\begin{frame}{obr}
 %   \begin{figure}[h!]
%	    \centering
%	    \includegraphics[width=0.9\textwidth]{nic.jpg}
%	    \caption{obr}
%	    \label{obr:obr}
 %   \end{figure}
%\end{frame}


\end{document}
